\documentclass{memoir}
\title{\textsc{In Search of a Liberal College}\\ A Program for the Recovery of The Classics and The Liberal Arts}
\author{Scott Buchanan\\ Dean and Fellow of St.\ John's College}
\date{September, 1937}
\begin{document}
\maketitle

\epigraph{It is a hard saying, but Mr.\ Eliot, more than any
other man, is responsible for the greatest educational crime
of the century against American youth---depriving him of his
classical heritage.}{\emph{Three Centuries of Harvard}\\
\textsc{Charles Eliot Morison}}

\tableofcontents*

\chapter{The Monastic Complex}

It seems that modern institutions and institutional attitudes are
related to the monastery as Freudian man is related to his mother
in the Oedipus complex. The person who openly enters a monastery is
frowned upon as a weakling who cannot face the world, as a psychopath
who seeks an institutional excuse for his solitary vices, or as an
irresponsible mystic who prefers to discharge his vital obligations
through otherworldly visions rather than through worldly deeds. These
attitudes are mirrored in the popular opinions that the monastery is
a vicious and corrupt institution, a romantic garden of moonlight and
fountains, or the typical and extreme example of fake asceticism. And
yet our reformers from the social settlement worker to the communist
commissar persistently and unconsciously revert to the monastic pattern
as the ideal and blueprint of their efforts. The school, the hospital,
the factory, and the model town continue the monastic theme, both by
mimicry and by protest. The social and cultural values that have lodged
in our consciences are monastic derivatives.

This is the clinical picture of a complex, it suggests an historical
perspective, and indicates a psychoanalytic cure through historical
and philosophical research. My attention was called to it by Lewis
Mumford's study of \emph{Technics and Civilization}. He there makes the
Benedictine monastery the vanishing point and origin of the industrial
revolution by calling attention to the fact that the mechanical clock
was invented by a Benedictine monk. The sailing vessel, the water-wheel,
and the windmill determine the major lines of the perspective, and
the later developments of the school, the hospital, the factory, and
the laboratory, are easy to place and fill in. The university and the
college fill the central foreground of the picture.

The secret of the maternal fecundity of this institution, Mr.\ Mumford
says, is the combination of the intellectual and the workman is each
member of the brotherhood. For the first time in European history the
division of labor between brain and brawn was balanced by a daily,
monthly, and yearly rotation of statuses and duties. This is epitomized
in the monastic day. It was divided into three parts, approximately
equal; not counting sleep, one-third was given to manual work, one-third
to bookish learning, and one-third to divine offices. Each brother
passed through this daily orbit, contributed his product, and received
his discipline. If Mr.\ Mumford is right this division of labor has
not been equalled in fruitfulness in European history, and our present
hesitations and frustrations in social ordering, in industrial planning,
in intellectual professions, and in matters of faith and morals have more
than an accidental connection with our bad monastic consciences.

It seems that the monastery, far from being an escape from life and the
world, was a courageous facing of life within an intelligible scheme
which might be imposed on chaos. In practical terms, the monastery
proposed and followed an ordering of the arts of life, the useful arts,
the liberal arts, and the divine arts. It withdrew from the wilder
and less workable regions of nature, relied on reason to construct
an intelligible human world from which with virtue and skill and
understanding the modern conquest of nature might proceed. This, I
take it, is a good description, if not a definition of an educational
institution, an institution of general human education. As a matter of
historical fact, it is the origin of our educational institutions of all
kinds.

The monastic project was born of a conception of man's nature. Man was
conceived as a creature between angels and animals. If he had been an
angel, his understanding would have been intuitive, immediate, and
complete; his actions would have been swift, accurate, and effective. If
he had been an animal, his understanding would have been a matter of
observation, trial and error, and operational; his actions would have
been instinctive, limited by sensations and feeling, uninformed by
imagination, unguided by foresight and principle. The monastery was set
up for a rational animal, having the possibility of intuitive insight
and understanding, provided his animal powers could be brought into
agreement and aid by training, discipline, and practice in the light of
idea and principle. The monastery accomplished its aim by a rotation of
practices in all the arts. We attempt a similar rotation by educational
reforms, experiments, and projects which pass from angel to animal, from
useful to liberal and divine arts, by sudden desperate jumps. Like the
now familiar electron, we do not know where we are if we know what we
are doing, and if we know where we are, we do not know what we are
doing. We are inquisitive and credulous of heresies because we do not
know where orthodoxy is to be found; we do not even want to be reminded
of the words heresy and orthodoxy because they remind us of an integrity
we fail to achieve.
\chapter{Tradition}

\chapter{The Classics}
\chapter{The Liberal Arts}
\chapter{The Laboratory Arts}
\end{document}
