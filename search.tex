\documentclass{memoir}
\title{\textsc{In Search of a Liberal College}\\ A Program for the Recovery of The Classics and The Liberal Arts}
\author{Scott Buchanan\\ Dean and Fellow of St.\ John's College}
\date{September, 1937}
\begin{document}
\maketitle

\epigraph{It is a hard saying, but Mr.\ Eliot, more than any other man,
is responsible for the greatest educational crime of the century against
American youth---depriving him of his classical heritage.}{\emph{Three
Centuries of Harvard}\\ \textsc{Charles Eliot Morison}}

\tableofcontents*

\chapter{The Monastic Complex}

It seems that modern institutions and institutional attitudes are
related to the monastery as Freudian man is related to his mother
in the Oedipus complex. The person who openly enters a monastery is
frowned upon as a weakling who cannot face the world, as a psychopath
who seeks an institutional excuse for his solitary vices, or as an
irresponsible mystic who prefers to discharge his vital obligations
through otherworldly visions rather than through worldly deeds. These
attitudes are mirrored in the popular opinions that the monastery is
a vicious and corrupt institution, a romantic garden of moonlight and
fountains, or the typical and extreme example of fake asceticism. And
yet our reformers from the social settlement worker to the communist
commissar persistently and unconsciously revert to the monastic pattern
as the ideal and blueprint of their efforts. The school, the hospital,
the factory, and the model town continue the monastic theme, both by
mimicry and by protest. The social and cultural values that have lodged
in our consciences are monastic derivatives.

\chapter{Tradition}
\chapter{The Classics}
\chapter{The Liberal Arts}
\chapter{The Laboratory Arts}
\end{document}
